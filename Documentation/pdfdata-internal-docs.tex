Due to the nature of its use, most of the macros defined in the pdfData section have already been explained.
The only part of pdfData which requires explanation regarding its internals is mappings, which offers richer features than already explained.

\subsection{Mappings}

Mappings are stored in two places: a {\it key list}, which is simply a macro consisting of pairs of the form \inlinecode|{key}{value}|, and macros \inlinecode|\key@k| (the second {\tt k} is variable in the name) whose definition
is {\tt v}.

Essentially, the major macro in this part is \macro\_mapkeys_with_setter\anchormacro\_mapkeys_with_setter.
Its usage is

\getmacrousage{\_mapkeys_with_setter <mapkey macro><key macro>{<map>}}
where {\it mapkey macro} is the macro which manages the creation of a key-value pair (explained below), {\it key macro} is a macro to store
the list of keys, and {\it map} is a map of key-value pairs.

What happens is \macro\_mapkeys_with_setter{} will iterate over {\it map} and for every key-value pair $({\tt k,v})$ if the setter {\it mapkey macro} is \macro\M{} and {\it key macro} is \macro\K{}, it calls
\inlinecode|\M \K{k}{v}|.
This should (if \macro\M{} is defined properly) update \macro\K{} to include the pair $({\tt k,v})$.
Furthermore, it should store the value {\tt v} in the macro \inlinecode|\key@k| (the second {\tt k} is variable in the name).

The macro \macro\_update_lastkeys{} is provided for the former: to update \macro\K{}.
Simply pass \inlinecode|\_update_lastkeys \K{k}{v}|.
The simplest setter ({\it mapkey macro}) is \macro\_vanilla_mapkey\anchormacro\_vanilla_mapkey, which does exactly what was described and nothing more.
Its definition is simply:

\blisting
\def\_vanilla_mapkey#1#2#3{%
    \_xp\def\csname key@\_id#2\endcsname{#3}%
    \_update_lastkeys{#1}{#2}{#3}%
}
\elisting

You can use the macro \macro\getvalue\anchormacro\getvalue{} to get the value of a key: its definition is simply

\blisting
\def\getvalue#1{%
    \csname key@#1\endcsname%
}
\elisting

Another macro is \macro\keyexists{} whose use is

\getmacrousage{\keyexists {<key>}<macro><key list>}
It checks if the key {\it key} is in {\it key list}, and if it is, defines {\it macro} to be equal to the key.
Otherwise {\it macro} is defined to be \macro\_nul.
For this reason, if you'd like a key to have no value, it is advised to use the \macro\novalue{} macro (whose definition is just \macro\novalue).

Another setter is \macro\_vardef_mapkey\anchormacro\_vardef_mapkey, whose only difference from \macro\_vanilla_mapkey{} is that instead of \macro\def ing \inlinecode|\key@k| to be equal to {\tt v}, \macro\_vardef_mapkey{} uses
\macro\_vardef{} instead of \macro\def{} (which can be set before calling \macro\_vardef_mapkey), and \macro\_vardef s \inlinecode|\key@k| to be the (once) expansion of \inlinecode|\_varmap{v}| (where \macro\_varmap) can also be
set before calling (\macro\_vardef_mapkey).

\macro\mapkeys{} is defined as follows:

\blisting
\def\mapkeys#1#2{%
    \_mapkeys_with_setter\_vanilla_mapkey\_keymappings{#1}%
    \_xp\_setdefaults\_xp{\_keymappings}%
    \_mapkeys_with_setter\_protected_mapkey\lastkeys{#2}%
    \_check_required_supplied%
}
\elisting

So first it gets the key-value pairs in {\it options} (\inlinecode|#1|) using \macro\_vanilla_mapkey; it places the results in \macro\_keymappings.
Then it sets the default values (this is what \macro\_setdefaults{} does; as well as figuring out which keys are required).
Then \macro\mapkeys{} calls \gotomacro\_mapkeys_with_setter{} using the setter \gotomacro\_protected_mapkey{} on {\it input} (\inlinecode|#2|).
It stores the results in \macro\lastkeys\anchormacro\lastkeys.
Then it checks that the required keys have been supplied (\macro\_check_required_supplied).

The setter \macro\_protected_mapkey\anchormacro\_protected_mapkey{} is more complicated than the previously-discussed setters.
Its use, like all setters, is

\getmacrousage{\_protected_mapkey <key list>{<key>}{<value>}}

But in this case, {\it key} has a value also in \macro\_keymappings{} as well; this value corresponds to another map containing the settings of {\it key} ({\tt name}, {\tt default}, {\tt required}, etc.).
So now \macro\_protected_mapkey{} will find the settings of {\it key}, and get the values of each field (via \macro\_mapkeys_with_setter).
Then it calls \macro\_vardef_mapkey{} with {\it key} and {\it value}, using the definitions of \macro\_vardef{} and \macro\_varmap{} according to the settings.
Finally it sets the macro {\tt name} (if provided in the settings) to be equal to the value.

