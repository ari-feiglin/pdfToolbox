The pdfGraphics section of the \pdftoolbox{} toolbox is for pdf-specific graphics macros.
You can use it to create colorful documents with illustrations, etc.

\subsection{Colors}

In {\tt pdfGraphics/colors.tex}, \pdftoolbox{} provides macros for coloring text and areas of your document.

\macroexp{\color <color space>{<code>}/cr\color {<name>}}
Switches the color of the document.
In its first form, {\it color space} corresponds to either {\tt rgb} or {\tt cmyk}, and {\it code} is either an rgb or cmyk code.
In its second form, if {\it name} is a predefined color name (see \gotomacro\definecolor), the color is switched to it.
\emacroexp

\macroexp{\localcolor <color space>{<code>}{<text>}/cr\localcolor {<name>}{<text>}}
Switches the color of {\it text}, according to the options provided (see \gotomacro\color).
\emacroexp

\macroexp{\definecolor {<name>}{<color space>}{<code>}}
Defines a color of name {\it name} whose space is {\it color space} (either {\tt rgb} or {\tt cmyk}) of code {\it code} (either an rgb or cmyk code).
\emacroexp

\macroexp{\letcolor {<new name>}{<name>}}
Defines a color of name {\it new name} to be equal to the existing color of name {\it name}.
\emacroexp

\macroexp{\definecolormacro {<name>}{<color space>}{<code>}}
Calls \gotomacro\definecolor, and also defines a macro of name {\it name} which is equivalent to \inlinecode|\localcolor <color space>{<code>}{#1}|.

The following colors are defined:

\centerline{\red{red} \blue{blue} \green{green} \yellow{yellow} \orange{orange} \purple{purple} \highlightbox{black}{\white{white}} \black{black} \darkgreen{darkgreen} \grey{grey}}
\emacroexp

\macroexp{\highlightbox <color space>{<code>}{<material>}
\highlightbox {<name>}{<material>}}
Colors the background of the material {\it material} according to the color provided.
For example \inlinecode|\highlightbox {red}{pdfToolbox}| will yield \highlightbox{red}{pdfToolbox}.
\emacroexp

\macroexp{\coloredbox <color space>{<code>}{<material>}
\coloredbox {<name>}{<material>}}
Like \gotomacro\highlightbox{} but adds a buffer of space around {\it material} in accordance with \macro\bufferwidth\anchormacro\bufferwidth{} and \macro\bufferheight\anchormacro\bufferheight.
For example \inlinecode|\coloredbox {red}{pdfToolbox}| will yield \coloredbox{red}{pdfToolbox}.
\emacroexp

\macroexp{\framecoloredbox <color space>{<code>}{<material>}
\framecoloredbox {<name>}{<material>}}
Like \gotomacro\coloredbox{} but adds a frame around {\it material} of width \macro\framewidth.
For example \inlinecode|\framecoloredbox {red}{pdfToolbox}| will yield \framecoloredbox{red}{pdfToolbox}.
\emacroexp

\macroexp{\framebox {<material>}}
Adds a frame around {\it material} with a buffer of \macro\bufferwidth{} and \macro\bufferheight{} of width \macro\framewidth.
\emacroexp

\macroexp{\curvedcolorbox {<stroke color>}{<bg color>}{<material>}{<curve control>}}
Creates a curved color framed box around {\it material} with frame color {\it stroke color} and background color {\it bg color} (which may be names or of the form \inlinecode|<color space>{<code>}|.
The curve's stroke width is determined by \macro\curvewidth, and the buffer around the material is determined by \macro\curvebuffer.
{\it control} is a sequence of $4$ symbols (either {\tt .} or {\tt X}) which determine whether a corner is curved or not.
A {\tt .} corresponds to a curve and a {\tt X} corresponds to a right corner.
A shadow of color \macro\boxshadowcolor{} is added to to the box, at an x and y offset of \macro\shadowxoff{} and \macro\shadowyoff{} respectively.

So for example:

\displaycode{%
\curvedcolorbox{blue}{red}{\color{white}pdfToolbox}{....}&\cr
\curvedcolorbox{blue}{red}{\color{white}pdfToolbox}{X...}&\cr
\curvedcolorbox{blue}{red}{\color{white}pdfToolbox}{.X..}&\cr
\curvedcolorbox{blue}{red}{\color{white}pdfToolbox}{..X.}&\cr
\curvedcolorbox{blue}{red}{\color{white}pdfToolbox}{...X}&}
\emacroexp

\macroexp{\fakebold {<material>}}
Bolds the material {\it material} (essentially just thickening the stroke width according to \macro\fakeboldwidth).
\emacroexp

\macroexp{\flip {<material>}}
\quitvmode\flip{flips} {\it material} about its vertical axis.
\emacroexp

