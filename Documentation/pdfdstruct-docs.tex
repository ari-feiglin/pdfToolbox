The pdfDstruct section of the \pdftoolbox{} toolbox is for managing the structure of your documents.

\subsection{Layout}

In {\tt pdfDstruct/layout.tex}, \pdftoolbox{} provides a macro \macro\setlayout\anchormacro\setlayout{} for setting up the layout of the document.
The use is

\macrousage\setlayout {[page width=<wd>,] [page height=<ht>,] [horizontal margin=<mwd>,]
[vertical margin=<vwd>]}/emacrousage

\subsection{Hyperlinks}

In {\tt pdfDstruct/hyperlinks.tex}, \pdftoolbox{} provides macros for creating and managing hyperlinks.

\macroexp{\anchor [<type>]{<name>}}
Creates an anchor (a reference, if you will) to the current point in the document.
\emacroexp

\macroexp{\gotoanchor [<type>]{<name>}{<material>}}
Creates a clickable field containing {\it material} which, when clicked, will go to the anchor labeled with the type {\it type} and name {\it name}.
\emacroexp

\macroexp{\url {<url>}{<material>}}
Creates a clickable field containing {\it material} which, when clicked, will redirect to the url {\it url}.
\emacroexp

\macroexp{\createbordertype {<type>}{<color>}{<wd>}}
Sets the border type of anchor type {\it type} to be of color {\it color} and width {\it wd}.
Urls have border type {\tt url}.
If a type doesn't have a specified border type, the {\tt default} one is used.
\emacroexp

\subsection{Fonts}

In {\tt pdfDstruct/fonts.tex}, \pdftoolbox{} provides macros for accessing and controlling fonts.

\macroexp{\addfont {<name>}{<sizes>}}
This will add a font by the name {\it name} so that it is accessible by \pdftoolbox.
{\it sizes} is a key-value dictionary which specifies the font codes for different sizes of the font.
For example, in {\tt pdfDstruct/fonts.tex} is the usage:

\blisting
\addfont{rm}{%
    default=cmr10,
    5pt=cmr5,
    6pt=cmr6,
    7pt=cmr7,
    8pt=cmr8,
    9pt=cmr9,
    10pt=cmr10,
    12pt=cmr12,
    17pt=cmr17
}
\elisting

So now \pdftoolbox{} has access to the computer modern roman font ({\tt cmr}) at the sizes specified.
The purpose of the default size is for when a size is not available.
For example, requesting the {\tt rm} font at size $13$ will give you {\tt cmr10 at 13pt}.
The default size is required.

\pdftoolbox{} provides the following fonts:

\emacroexp

\quitvmode\vbox{\tabskip=.5cm\halign{&\hfil#:\tabskip=0pt&\ \tt#\hfil\tabskip=.5cm\cr
rm&cmr&it&cmti&bf&cmbx&sc&cmcsc&mi&cmmi&sy&cmsy&ex&cmex&sl&cmsl\cr
ss&cmss&tt&cmtt&msam&msam&msbm&msbm&eufm&eufm&rsfs&rsfs\cr
}}

\macroexp{\applyfontcode <font code>}
Applies the font specified by {\it font code}.
For example, \inlinecode|\applyfontcode cmr10| will set the font to {\tt cmr10}.
\emacroexp

\macroexp{\setfontfamily {<font>}{<family>}}
Sets math font family {\it family} to the font {\it font} (which is specified by \gotomacro\addfont).
For example, \inlinecode|\setfontfamily{rm}{0}| sets the alpha-numeric font family to {\tt rm}.
\emacroexp

\macroexp{\setfont {<font>}}
Sets the current font to {\it font}.
The current font is stored in the macro \macro\currfont\anchormacro\currfont.
\emacroexp

\macroexp{\setscale {<scale>}}
Sets the current font scale to {\it scale}.
The current font scale is stored in the macro \macro\currscale\anchormacro\currscale.
\emacroexp

\macroexp{\setfontandscale {<font>}{<scale>}}
Sets the current font to {\it font} and scale to {\it scale}.
\emacroexp

\pdftoolbox{} also provides the following font switches (which are simple wrappers around \macro\setfont{} which also set \macro\fam):

\hbox to\hsize{\hfil\inlinecode|\bf, \it, \bb, \sf, \sl, \frak, \scr|\hfil}

\macroexp{\mathfonttable {<family>}[<offset>]{<table>}}
The \macro\mathfonttable{} macro's purpose is to define multiple mathematical characters for the same family.
{\it table} consists of a sequence of macros followed by numbers (e.g. \inlinecode|\square0|) which correspond to the name of the macro and the math type (in this case $0$: ordinary/\macro\mathord).
\macro\mathfonttable{} will iterate over {\it table} and \macro\mathchardef{} the macro to be equal to the character at the current position in family {\it family} of the type specified.
If {\it offset} is specified, it will start iterating over the family starting from the offset.

More explicitly, if {\it family} is {\tt X} and the $\tt i$th index in the table is \inlinecode|\X N|, then the macro does essentially

\hbox to\hsize{\hfil\inlinecode|\mathchardef\X = XNi|\hfil}

To skip over an index, simply write \macro\__{}.
\emacroexp

Using \macro\mathfonttable, \pdftoolbox{} defines the following:

{\tabskip=2pt\halign{&\hfil\def\X{#}\_xp\macro\X: $\X$\hfil\cr
    \boxdot&\boxplus&\boxtimes&\square\cr
    \blacksquare&\diamond&\blackdiamond&\rotateclockwise\cr
    \rotatecounterclockwise&\rightleftharpoons&\leftrightharpoons&\boxminus\cr
    \Vdash&\Vvdash&\vDash&\twoheadrightarrow\cr
    \twoheadleftarrow&\leftleftarrows&\rightrightarrows&\upuparrows\cr
    \downdownarrows&\uprightharpoon&\downrightharpoon&\upleftharppon\cr
    \downleftharpoon&\rightarrowtail&\leftarrowtail&\leftrightarrows\cr
    \rightleftarrows&\Lsh&\Rsh&\rightsquigarrow\cr
    \leftrightsquigarrow&\looparrowleft&\looparrowright&\circeq\cr
    \succsim&\gtrsim&\gtrapprox&\multimap\cr
    \therefore&\because&\Doteq&\triangleq\cr
    \precsim&\lesssim&\lessapprox\cr
}}

\subsection{Hooks}

\pdftoolbox{} provides a tool, inspired by \LaTeX, called {\it hooks} (source in {\tt pdfDstruct/hooks.tex}).
Hooks are simply snippets of code that can be inserted into macros and then altered later.
An example is given at the end of this section.

\macroexp{\createhook {<name>}}
Creates a hook by the name of {\it name}.
\emacroexp

\macroexp{\appendtohook {<name>}{<code>}}
Appends {\it code} to the hook specified by {\it name}.
\emacroexp

\macroexp{\prependtohook {<name>}{<code>}}
Prepends {\it code} to the hook specified by {\it name}.
\emacroexp

\macroexp{\callhook {<name>}}
Calls the hook specified by {\it name}.
\emacroexp

\pdftoolbox{} provides a builtin hook called {\tt end} which is executed by \macro\bye.
Throughout the document, you can add macros to an array called {\tt document data}, then all these definitions are written to the file \inlinecode|\jobname.data| by the {\tt end} hook.

Specifically, you can use the \macro\docdata\anchormacro\docdata{} macro to add a macro to the document's data, e.g. if you have a macro \macro\name{} which has the author's name (say, S. Lurp), you can do
\inlinecode|\docdata\name|, and this will write the line \inlinecode|\gdef\name{S. Lurp}| to the data file.
Then at the beginning of the document next compilation, you can load all definitions in the data file.

\subsection{Indices}

In {\tt pdfDstruct/index.tex}, \pdftoolbox{} provides macros for creating an index.
The index is organized into {\it categories} and {\it items} within each category, and an associated {\it value}.
A category may be something like ``manifolds'' and an item within this category may be ``topological'' which has a value corresponding to the page number where topological manifolds are defined.

\macroexp{\indexize {<options>}}
Adds an item to the index, specified by options, which has fields:
\benum
    \item {\tt category} (required): the category of the item;
    \item {\tt item}: the item of the item;
    \item {\tt value} (required): the value of the item;
    \item {\tt expand value} (valueless): added if {\tt value} should be expanded (e.g. if {\tt value} is a macro corresponding to the page number, it needs to be expanded);
    \item {\tt add hyperlink} (valueless): whether or not the item's values should be hyperlinked.
\eenum
\emacroexp

\macroexp{\seealso {<options>}}
Adds a ``see also'' item to the index: one which redirects to another index item.
{\it options} is a map which has fields:
\benum
    \item {\tt category} (required): the category of the item;
    \item {\tt item}: the item of the item;
    \item {\tt dest} (required): the destination of the ``see also'' (e.g. if the item is ``wedge product'', you may want to also see ``exterior product'', and so the destination may be ``exterior product'');
    \item {\tt hyperlink}: an anchor to link to;
    \item {\tt index link} (valueless): a flag of whether or not the anchor is within the index.
\eenum
To link to an item within the index, suppose of category {\tt C} and item {\tt I}, set {\tt hyperlink} to {\tt C:I} (or just {\tt C:} if {\tt I} is empty), and set {\tt index link}.
\emacroexp

\macroexp{\index}
Prints the index.
\emacroexp

\macroexp{\addtoindex {<category>}[<item>]}
Adds an item to the index of category {\it category} and item {\it item}.
Its value is \macro\@defaultindexval{} (by default \inlinecode|\the\pageno|), and {\tt expand value} and {\tt add hyperlink} are set.
\emacroexp

\subsection{Lists}

In {\tt pdfDstruct/lists.tex}, \pdftoolbox{} provides macros for creating lists of text.

There are two types of lists: unenumerated and enumerated.
Unenumerated lists start with \macro\blist\anchormacro\blist{} and end with \macro\elist\anchormacro\elist.
Each item begins with \macro\item\anchormacro\item.
The symbol used for each bullet point is determined by the nested depth of the list.
For a depth of $N$, the symbol used is stored in the macro \macro\liststyleN\anchormacro\liststyle.

Similarly enumerated lists start with \macro\benum\anchormacro\benum{} and end with \macro\elist\anchormacro\eenum.
Each item begins with \macro\item, and the style for the enumeration is determined by the depth of the list.
For a depth of $N$, the $n$th element is styled with \inlinecode|\enumstyleN{n}|.

To add text in between items (not as part of the list), you can use \macro\mtext\anchormacro\mtext.

\subsection{Table of Contents}

In {\tt pdfDstruct/tableofcontents.tex}, \pdftoolbox{} provides macros for creating and displaying tables of content.

\macroexp{\addtoccontent {<marker>}{<title>}{<value>}{<depth>}{<anchor>}}
Adds content to the table of contents.
The marker is {\it marker} (e.g. $1.1$; this is printed to the left of the title), title is {\it title} (e.g. chapter name), value is {\it value} (e.g. page number), depth is {\it depth},
and is linked to the anchor {\it anchor}.
The depth {\it depth} determines the style used in the table (see \gotomacro\settocdepthformat).
\emacroexp

\macroexp{\tableofcontents}
Prints the table of contents.
\emacroexp

\macroexp{\settocdepthformat {<depth>}{<options>}}
Sets the format of the table of contents at the depth {\it depth}.
{\it options} is a map with the following fields:
\blist
    \item {\tt marker}: the style for the marker (default is \inlinecode|\setfont{rm}|; the marker is passed as a parameter to {\tt marker});
    \item {\tt marker buffer}: the buffer between the title and marker (default is {\tt.25cm});
    \item {\tt title}: the style for the title (default is \inlinecode|\setfont{rm}|; the title is passed as a parameter to {\tt title});
    \item {\tt value}: the style for the value (default is \inlinecode|\setfont{rm}|; the value is passed as a parameter to {\tt value});
    \item {\tt leader}: the leader to add between the title and value (default is nothing);
    \item {\tt indent}: the amount to indent the line (default is 0pt);
    \item {\tt buffer}: the amount of buffer to add around the line (default is 0pt).
\elist
\emacroexp

\pdftoolbox{} provides four types of sectioning: parts, sections, subsections, and subsubsections.
Each has a counter in its name (e.g. {\tt section}), and a macro with the current section name (e.g. \macro\currsection).

\macroexp{\section (*){<title>}}
Adds a section to the document.
If the asterisk is added, the section is a ``pseudosection'': the section counter is not incremented and not displayed, and the section is not added to the table of contents.
Otherwise the section counter is incremented and displayed, and the section is added to the table of contents.
\emacroexp

\macroexp{\subsection (*){<title>}}
Adds a subsection to the document.
If the asterisk is added, the subsection is a ``pseudosubsection'': the subsection counter is not incremented and not displayed, and the subsection is not added to the table of contents.
Otherwise the subsection counter is incremented and displayed, and the subsection is added to the table of contents.
\emacroexp

\macroexp{\subsubsection (*){<title>}}
Adds a subsubsection to the document.
If the asterisk is added, the subsubsection is a ``pseudosubsubsection'': the subsubsection counter is not incremented and not displayed, and the subsubsection is not added to the table of contents.
Otherwise the subsubsection counter is incremented and displayed, but the subsubsection is still not added to the table of contents.
\emacroexp

