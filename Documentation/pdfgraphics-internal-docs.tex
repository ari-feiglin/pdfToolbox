\subsection{Colors}

There are some useful macros in the {\tt pdfGraphics/colors.tex}, here we describe them.

\bwarning
These macros and file require a clean-up.
Unfortunately many other macros are dependent on them, and I am scared to significantly alter anything.
One day, though.
\eppbox

\macroexp{\_rgb_encode {<rgb code>}/cr
\_rgb_encodebg {<rgb code>}/cr
\_rgb_encodefg {<rgb code>}/cr
\_cmyk_encode {<cmyk code>}/cr
\_cmyk_encodebg {<cmyk code>}/cr
\_cmyk_encodefg {<cmyk code>}}
Gets the code for the specified color for the foreground or background or both.
\emacroexp

\macroexp{\_setcolor_code {<pdf code>}}
Sets the current color using {\it pdf code} (which can be obtained using one of the above macros).
Essentially just pushing {\it pdf code} onto the color stack.
After the current group, calls \gotomacro\_pdfcolor_restore.
\emacroexp

\macroexp{\_pdfcolor_restore}
Restores the color (pops from the color stack).
\emacroexp

\macroexp{\_color_set {<color space>}{<color code>}/cr
\_colorbg_set {<color space>}{<color code>}/cr
\_colorfg_set {<color space>}{<color code>}}
Sets the current color using {\it color code} according to {\it color space} (either {\tt rgb} or {\tt cmyk}).
\emacroexp

\macroexp{\_color_defined {<name>}/cr
\_colorbg_defined {<name>}/cr
\_colorfg_defined {<name>}}
Sets the current color according to the color {\it name} (see \gotomacro\definecolor).
\emacroexp

\macroexp{\_getcolorparam <macro>{<place>}<color>}
Gets the pdf code for {\it color} (which may be of the form \inlinecode|rgb{...}|, \inlinecode|cmyk{...}|, or \inlinecode|{name}|), and calls {\it macro} with it as a parameter.
{\it place} is either {\tt fg}, {\tt bg}, or left empty.
\emacroexp

\macroexp{\_setcolor {<place>}{<color>}}
Sets the current color according to {\it place} and {\it color}.
{\it place} is either {\tt fg}, {\tt bg}, or left empty.
\emacroexp

\macroexp{\_getcolor {<place>}{<color>}}
Expands to the pdf code for {\it color} ({\it place} is either {\tt fg}, {\tt bg}, or left empty).
\emacroexp

\subsection{Colorboxes}

\pdftoolbox{} provides a relatively simple interface for creating colorboxes like \macro\bppbox.
The main macro is \macro\_splitcontentbox\anchormacro\_splitcontentbox, whose usage is

\getmacrousage{\_splitcontentbox {<buffer>}<macroT><macroS><macroM><macroE>}

Which repetitively splits the box \macro\_contentbox\anchormacro\_contentbox{} into \macro\_splitbox{} to fill the remaining material on a page or in the box itself.
Then the split box is passed to {\it macroX} for pretty formatting.
{\it macroT} is if the material fits entirely on a single page, otherwise the first box uses {\it macroS}, the
last box uses {\it macroE}, and all intermediate boxes use {\it macroM}.
{\it buffer} is the total amount of vertical buffering that {\it macro} adds to the box it prints.

So to create your own prettyprint-box (ppbox), you create two macros, say \macro\beginpp{} and \macro\endpp.
In \macro\beginpp{} you add the code which should go before the ppbox and starts getting content for \macro\_contentbox.
For example, it could be as simple as:

\blisting
\def\beginpp#1#2{%
    \def\_colorcontentbox{%
        \hbox{\coloredbox{#1}{\_setcolor{}{#2}\box\_splitbox}}%
    }%
    \par\kern.5cm\null\par%
    \setbox\_contentbox=\vbox\bgroup
        \hsize=\dimexpr\hsize-\bufferwidth * 2\relax%
}

\def\endpp{%
    \egroup%
    \_splitcontentbox{\bufferwidth * 2}%
        \_colorcontentbox\_colorcontentbo\_colorcontentbo\_colorcontentboxxx%
    \kern.5cm\relax%
}
\elisting

This creates a ppbox which is simply a wrapper around \gotomacro\coloredbox.
It colors the background in \inlinecode|#1| and the foreground in \inlinecode|#2|.

In depth, here's how it works:
\benum
    \item First, \macro\beginpp{} defines \macro\_colorcontentbox{} to simply place \macro\_splitbox{} into a \macro\coloredbox{} of color \inlinecode|#1|, and sets the foreground color to \inlinecode|#2|.
    \item Then it adds some space before the start of the first ppbox.
        The reason for the \inlinecode|\null\par| is to move the kern from the list of recent contributions to the main vertical list (see, e.g. the \TeX book for more information on \TeX's output routines).
    \item Then \macro\beginpp{} begins reading content for \macro\_contentbox.
        It alters \macro\hsize{} to compensate for the buffer added by \macro\coloredbox.
    \item When \macro\endpp{} is called, it first stops the capture of \macro\_contentbox{} with \macro\egroup.
    \item Then it calls \inlinecode|\_splitcontentbox{\bufferwidth * 2}\_colorcontentbox|, which splits the captured material (in \macro\_contentbox) and places each \macro\_splitbox{} in \macro\_colorcontentbox, which was
        defined in \macro\beginpp.
        \inlinecode|\bufferwidth * 2| corresponds to the amount of vertical buffering \macro\_colorcontentbox{} adds to \macro\_splitbox.
    \item \macro\endpp{} adds buffering after the final ppbox.
\eenum

\subsubsection{The Mechanism}

In this section we describe the mechanism through which \pdftoolbox{} creates colorboxes.
A good reference for this section, which discusses the mechanism through which \TeX{} creates pages (the output routine) is ``The Advanced \TeX book'' by David Salomon.

We begin with a simple macro which expands to the amount of space left in the page.
\listfile{ptb-colorboxes}[9-15]
\macro\pagetotal{} is the height of the main vertical list (MVL).
\macro\pagegoal{} is the goal height of the MVL.
It is generally equal to \macro\vsize, but when footnotes or similar are generated, their heights are
subtracted from \macro\pagegoal.
At the beginning of each page, \macro\pagegoal{} is set to \macro\maxdimen{} (hence the use of the
\macro\ifdim).

\pdftoolbox{} stores the content to be split across pages in \inlinecode|\box\_contentbox|.

The following macro \macro\_getsplitdim{} gives the dimension that \macro\_contentbox{} should be split into.
Its parameter is the size of the extra vertical glue that is added to each box.
\listfile{ptb-colorboxes}[17-41]
\benum
    \item First we check if there is enough space to place the entirety of the box into the page.
        If so set \macro\_splitdim{} to the height of the box, and set \macro\_lastbox{} to true since we wil be placing the entirity of the box.
    \item Otherwise, check if there is any space left on the current page (recall that \inlinecode|#1| is the
        amount of vertical space added.)
        If there is, then set \macro\_splitdim{} to the amount of space left.
    \item Otherwise, we perform the following checks:
        \benum
            \item If \macro\_recheck{} is false, then we add \inlinecode|\null\par| to the page.
                This just moves all recent contributions to the MVL.
                The reason for this is that we need to get rid of all the material that came before the
                colorbox, and we then recheck the dimension, and set \macro\_recheck{} to true.
            \item Otherwise, the MVL is up-to-date, and there is still not enough room.
                So we try to fill in the rest of the vertical space and \macro\break.
                If for whatever reason this doesn't work, we add \macro\vfilneg{} to remove the glue added.
        \eenum
        In either case we get the dimension again.
\eenum

Now, the main macro is \macro\_splitcontentbox:
\listfile{ptb-colorboxes}[46-77]
It first sets \macro\splittopskip{} to $0{\tt pt}$ so that no extra glue is added to the top of \macro\vsplit.
Then \macro\boxmaxdepth{} is also set to $0{\tt pt}$ so that the depth of the split boxes will be $0{\tt pt}$
and we can deal only with height.
We also turn off interlineskip so there is no extra glue added around the split boxes (these do not affect
the contents of \macro\_contentbox{} since it has already been boxed).

Now we repeat until \macro\_contentbox{} is empty:
\benum
    \item we get the amount of space to split the box into via \macro\_getsplitdim{} (explained above);
    \item we split \macro\_contentbox{} into \macro\_splitbox{} of this dimension;
    \item \inlinecode|#2| is the output routine of the colorbox, it places the contents of \macro\_splitbox{}
        into whatever format the user specifies.
        We set \inlinecode|\box0| to this;
    \item we add the box to the page, and set a penalty of $0$ so that the page can be broken at that point
        if necessary.
\eenum

The definition of \macro\bppbox{} and \macro\eppbox{} are a little enlightening:
\listfile{ptb-colorboxes}[79-116]
What's of interest here is how \macro\_bppbox{} defines the different \macro\_colorcontentbox s.
Firstly, \macro\_getdotsnlines{} is a macro which defines \macro\_line{\it side} and \macro\_dot{\it corner} according to the stream of $8$ characters which follow it (until \macro\_nul).
Now, \macro\_colorcontentboxT{} is defined as you'd expect.
And \macro\_colorcontentboxS{} is defined so that it preserves the top corners that are input as well as all but the bottom side.
\macro\_colorcontentboxM{} and \macro\_colorcontentboxE{} are defined similarly.

\subsection{Illustrating} \anchor{internal-pdfdraw}

\bwarning
This is a complicated and messy part of \pdftoolbox.
Documentation will be added once it is cleaned up.
\eppbox

\subsection{Listings} \anchor{internal-listings}

\subsubsection{The Mechanism}

We first begin by discussing the mechanism for how listings work in \pdftoolbox.
Credit where credit is due; the mechanism is greatly inspired and copied from Petr Ol\v s\'ak's \url{https://ctan.org/pkg/optex?lang=en}{Op\TeX}, though the implementation may differ.

The mechanism is relatively simple: all \pdftoolbox{} does is the following:
\benum
    \item call \inlinecode|\the\everylisting| and whatever is given in the remaining line after \macro\blisting;
    \item call \macro\setupverb{} which changes the catcode of special characters and \inlinecode|^^M| to $12$;
    \item capture the entirety of the listing from \macro\blisting{} to \macro\elisting;
    \item set line spacing (via \macro\_setuplstlines);
    \item call \inlinecode|\the\_commandcapture| which simply sets up macros which are called in \inlinecode|\the\_listingcommands|, which is called immediately afterward (see below);
    \item call \inlinecode|\the\_commandexecute| which executes the commands added in \inlinecode|\the\_listingcommands| on the listing, and then \inlinecode|\the\_macrocallmanager| which alters the definitions of the macros in
        \macro\_listingcommands{} to their proper definitions (explanation later);
    \item at this point, the listing is set up so that everything is ready for printing;
    \item the line manager is set up (which handles printing each line in the listing);
    \item the additional vertical buffering added by \macro\syntaxoutbox{} is computed into \macro\syntaxoutboxbuf{} by \macro\syntaxoutboxsetbuf;
    \item the listing is processed through a ppbox dictated by \macro\syntaxoutbox.
\eenum

This is a deceptive description of how this process works.
But what's important is \macro\_commandexecute, which has all the commands for setting up syntax highlighting and the verbatim environment.
This is not a token list to messed with by the user directly, it should be done indirectly through \macro\_listingcommands, which in turn should be altered indirectly through ptb-syntax files.

But if no syntax is set, the definition of \macro\_commandexecute{} is essentially:

\blisting
\_commandexecute={
    \_execute{\_r_replace{ }{\w \w}}
    \_execute{\_r_replace{^^M}{\w\n\n\w}}
}
\elisting

\macro\_execute{} is defined simply to \inlinecode|\def\_execute#1{#1\_code}| right before execution, where \macro\_code{} is the captured listing.
And \macro\_r_replace{} is defined to simply be \inlinecode|\def\_r_replace#1#2#3{\replace{#3}{#1}{#2}}|.
So when \macro\_commandexecute{} is called, the result is simply two lines: \inlinecode|\replace{\_code}{ }{\w \w}| and \inlinecode|\replace{\_code}{^^M}{\w\n\n\w}|.

\macroexp{\replace <macro>{<pattern>}{<replacement>}}
Replaces (the expansion of) {\it pattern} with (the expansion of) {\it replacement} in the definition of {\it macro}.
\emacroexp

So the result is that now \macro\_code{} contains the listing, but where each space is now swapped with \inlinecode|\w \w| and each line ending with \inlinecode|\w\n\n\w|.
\macro\_code{} is actually defined to be the listing wrapped in \inlinecode|\n\w...\w\n|, so the result is that every line in the listing is wrapped in \inlinecode|\n...\n| and every word is wrapped in \inlinecode|\w...\w|.

Now suppose you wanted to replace all occurrences of {\tt hi} with {\tt hello}.
You'd need to add the line \inlinecode|\_execute{\_r_replace{\w hi\w}{\w hello\w}}| to \macro\_commandexecute{} (note that the pattern and replacement are wrapped in \macro\w s).
You can do so with the following command:

\macroexp{\_add_command_replace {<pattern>}{<replacement>}}
Adds \getmacrousage{\_execute{\_r_replace{<pattern>}{<replacement>}}} to \macro\_commandexecute.
\emacroexp

But this is unwieldy, so the actual mechanism used is as follows:
\benum
    \item \macro\_commandcapture{} sets the definitions of \macro\replace{} and \gotomacro\replacefromto{} (see below) to\hfill\break\macro\_add_command_replace{} and \macro\_add_command_replacefromto{}.
    \item \macro\_listingcommands{} contains uses of \macro\replace{} and \macro\replacefromto{}, which are executed.
        This adds the required lines to \macro\_commandexecute.
    \item \macro\_commandexecute{} is executed.
\eenum

Now to explain \gotomacro\loadsyntax{} and \gotomacro\setsyntax: \getmacrousageB{\loadsyntax {<language>}} reads from the ptb-syntaxptb-syntax  file {\tt ptb-syntax-{\it language}.tex}, which should define a token list
\macro\_{\it language}{\tt _listingcommands}.
Then \macro\setsyntax{} simply sets \macro\_listingcommands{} to \macro\_{\it language}{\tt _listingcommands}.

\subsection{Usage}

Now notice an issue: \macro\replace{} (and \macro\replacefromto) both expand their arguments.
What if the arguments expand to invalid code?
This is the purpose of \macro\_macrocallmanager{} and two basic macros: \macro\call{} and \macro\mlcall.
They are set to \macro\relax{} so they aren't expanded in \macro\replace{} and friends, and \macro\_macrocallmanager{} sets them to their proper definitions:

\macroexp{\call <macro>{<parameters>}{<last>}}
Simply calls \getmacrousageB{\macro <parameters>{<last>}}.
\emacroexp

\macroexp{\mlcall <macro>{<parameters>}{<last>}}
If {\it last} is equal to \inlinecode|x1\n\n x2\n\n...\n\n XN|, expands to \inlinecode|\call <macro>{<parameters>}{x1}\n\n...\n\n\call <macro>{<parameters>}{xN}|.
\emacroexp

Since \macro\call{} and friends are redefined only after \macro\_commandexecute{} is executed, to call a macro without it expanding you can use them.
For example, to replace {\tt int} with {\tt int} colored red, you can use \inlinecode|\replace{\w int\w}{\w\call localcolor{{red}}{int}\w}|.

Two shortcuts are provided: \macro\c{} and \macro\mc{}.
\getmacrousageB{\c{<color>}{<text>}} will set the color of {\it text} (in a \macro\call), and \getmacrousageB{\mc{<color>}{<text>}} will also set the color but in a \macro\mlcall.

Now what is \macro\replacefromto?

\macroexp{\replacefromto <macro>{<start>}{<end>}{<replacement>}}
Replacement is a macro definition with a single pattern (e.g. \inlinecode|{#1}|).
\macro\replacefrom{} matches \getmacrousage{<start>#1<end>} in the expansion of {\it macro} and replaces it with {\it replacement}, this redefines {\it macro}.

So for example if \macro\X{} expands to \inlinecode|(.)(.)|, \inlinecode|\replacefrom\X(){[#1]}| will redefine \macro\X{} to be \inlinecode|[.][.]|.
\emacroexp

\subsubsection{An Example}

\pdftoolbox{} provides syntax highlight for the C language in {\tt ptb-syntax-C.tex}, whose content is:

\listfile{ptb-syntax-C}

Let us now explain each part of the file:

\listfile{ptb-syntax-C}[1-9]

This defines two new token lists, \macro\lstCcolor{} (for use within the file), and \macro\_C_listingcommands,
which as explained previously is what sets the listing commands for the {\tt C} syntax.
\macro\lstCcolor{} has C-specific colors, changing it allows you to change the colors of C-specific
highlighting.
Note that all changes in this file must be global.

\listfile{ptb-syntax-C}[11-12]

Begins the definition of \macro\_C_listing_commands{} by first defining C-specific colors.

\listfile{ptb-syntax-C}[13-17]

\benum
    \item Swaps \macro\"{} with \inlinecode|{\"}|, so that the quotation in \macro\"{} (backslash-quote) is
        not replaced by the following lines.
    \item Colors between \inlinecode|"| and \inlinecode|"| with a multiline coloring of color {\tt quote}.
    \item Colors between \inlinecode|//| and \macro\n{} (the end of the line) with the color of a comment.
        Notice that it adds back in the \macro\n; otherwise this will mess up the line-reading.
    \item Replaces between \inlinecode|/*| and \inlinecode|*/| with a multline coloring of color
            {\tt lst-comment}.
    \item Replaces from \inlinecode|#| to the end of line with a coloring of color {\tt preprocessor}.
        (Again adding back in \macro\n.)
\eenum

\listfile{ptb-syntax-C}[18-22]

Sets replacement for \TeX-reserved characters (open and close curly braces, percent).

\listfile{ptb-syntax-C}[23-25]

Replaces non-character letters with a coloring and word break (since, e.g. {\tt x.y} is not a single word).
Note the use of \gotomacro\map.

\listfile{ptb-syntax-C}[26-32]

Replaces keywords with color {\tt keyword}.

\listfile{ptb-syntax-C}[33-34]

Colors numbers with {\tt lst-number}.

\subsubsection{Changing the Output}

\pdftoolbox{} outputs the listing in colorboxes according to \macro\syntaxoutbox.
The default is just as follows:

\listfile{ptb-listings}[121-127]

This just places the content in a colored box of color {\tt lst-bg} with a text color of {\tt lst-fg}, along with placing a header and footer.
We must also set \macro\syntaxoutboxbuf, which is the total amount of vertical buffering added by
\macro\syntaxoutbox{} to its contents.
This is done by \macro\syntaxoutboxsetbuf{} which must define a macro \macro\syntaxoutboxbuf{} to be the total amount of extra vertical space \macro\syntaxoutbox{} adds.

But we can also do, for example:

\blisting
\letcolor{lst-stroke}{lst-fg}
\def\syntaxoutbox#1{%
    \hbox{\curvedcolorbox{lst-stroke}{lst-bg}{\_setcolor{}{lst-fg}#1}{-.-.-.-.}}%
}
\def\syntaxoutboxsetbuf{%
    \edef\syntaxoutboxbuf{\the\dimexpr\_actual_curve_buf * 2\relax}%
}
\elisting

\bgroup
\letcolor{lst-stroke}{lst-fg}
\def\syntaxoutbox#1{%
    \hbox{\curvedcolorbox{lst-stroke}{lst-bg}{\_setcolor{}{lst-fg}#1}{-.-.-.-.}}%
}
\def\syntaxoutboxsetbuf{%
    \edef\syntaxoutboxbuf{\the\dimexpr\_actual_curve_buf * 2\relax}%
}

Now listings will have a curve colored box:

\blisting
\def\X#1#2{%
    \ifnum#1>0 %
        {\X{\numexpr#1-1\relax}{#2}}^{\X{\numexpr#1-1\relax}{#2}}_{\X{\numexpr#1-1\relax}{#2}}%
    \else%
        \vcenter{\hbox{$\m@th\scriptscriptstyle#2$}}%
    \fi%
}

$$\X{4}\blacksquare$$
\elisting
\egroup

(The code outputs the following by the way:)

\bgroup\catcode`_=8
\def\X#1#2{%
    \ifnum#1>0 %
        {\X{\numexpr#1-1\relax}{#2}}^{\X{\numexpr#1-1\relax}{#2}}_{\X{\numexpr#1-1\relax}{#2}}%
    \else%
        \vcenter{\hbox{$\m@th\scriptscriptstyle#2$}}%
    \fi%
}

$$\X{4}\blacksquare$$
\egroup

