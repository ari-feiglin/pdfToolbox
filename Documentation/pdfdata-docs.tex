The pdfData section of the \pdftoolbox{} toolbox is meant for creating instances of and manipulating datatypes.

\subsection{Arrays}

In the {\tt pdfData/arrays} file, \pdftoolbox{} defines various macros for creating and manipulating arrays.
There are two types of arrays, which are different in the macros used for them and the way they are stored internally.
\blist
    \item (Normal) arrays: these arrays are stored in the traditional way: an array $[1;2;3]$ is stored in a macro whose meaning is equivalent to \inlinecode|\X{1}\X{2}\X{3}|.
    Manipulation of the array is done by defining \macro\X, and then executing the array macro.
    \item Macro arrays: these arrays are stored in a collection of macros: each element is stored in its own indexed macro.
    So an array $[1;2;3]$ will be stored in three macros, whose values are $1$, $2$, $3$ respectively.
\elist

\bnote

    All arrays are zero-indexed.

\eppbox

\subsubsection{Normal Arrays}

\macroexp{\createarray {<name>}}
creates an (normal) array whose name is {\it name}.
\emacroexp

\macroexp{\ensurearray {<name>}}
ensures that an array by the name of {\it name} exists.
\emacroexp

\macroexp{\localizearray {<name>}}
localizes (see \gotoanchor{localization}{localization}) the array named by {\it name}.
\emacroexp

\macroexp{\appendarray {<name>}{<value>}}
appends {\it value} to the end of the array array named by {\it name}.
{\it value} is inserted according to \gotomacro\currdef.
\emacroexp

\macroexp{\prependarray {<name>}{<value>}}
prepends {\it value} to the end of the array array named by {\it name}.
{\it value} is inserted according to \gotomacro\currdef.
\emacroexp

\macroexp{\appendarraymany {<name>}{{<value1>}{<value2>}...{<valueN>}}}
appends {\it value1} through {\it valueN} to the end of the array array named by {\it name}.
Each {\it value} is inserted according to \gotomacro\currdef.
\emacroexp

\macroexp{\arraylen {<name>}}
expands to the length of the array specified by {\it name}.
\emacroexp

\macroexp{\getarraylen {<name>}<macro>}
inserts the length of the array specified by {\it name} into the macro {\it macro}.
\emacroexp

\macroexp{\arraymap {<name>}<macro>}
if the array specified by {\it name} is equivalent to $\tt [x0;\dots;xN]$ then doing \inlinecode|\arraymap{name}\X| will execute \inlinecode|\X{x1}{0}...\X{xN}{N}|.
\emacroexp

\macroexp{\indexarray {<name>}{<i>}<macro>}
Puts the $i$th element in the array specified by {\it name} into the macro {\it macro}.
\emacroexp

\macroexp{\removearray {<name>}{<i>}<macro>}
Removes the $i$th element in the array specified by {\it name} and places it into the macro {\it macro}.
\emacroexp

\macroexp{\removeitemarray {<name>}{<value>}}
Removes all instances of {\it value} from the array specifed by {\it name} (comparison is done using \macro\ifx{} on macros containing {\it value} and the current index).
\emacroexp

\macroexp{\printarray {<name>}}
Prints the array specified by {\it name}.
\emacroexp

\macroexp{\copyarray {<src>}{<dest>}}
Copies the array {\it src} into {\it dest}.
\emacroexp

\macroexp{\concatenatearrays {<arr1>}{<arr2>}{<dest>}}
Concatenates the arrays {\it arr1} and {\it arr2} and places the result into a new array {\it dest}.
\emacroexp

\macroexp{\initarray {<name>}{<x1>,...,<xN>}}
Creates a new array by the name of {\it name} equivalent to $\tt[x1;\dots;xN]$.
\emacroexp

\macroexp{\findarray {<name>}{<value>}}
Checks if the value {\it value} exists in the array {\it name} (checking is done via \macro\ifx).
If the value exists, the value \macro\True{} is placed into \macro\@return@value, otherwise it is equal to \macro\False.
\emacroexp

\macroexp{\uniqueappendarray {<name>}{<value>}}
Appends {\it value} to the array {\it name} only if it does not already exist in {\it name} (\macro\@return@value{} is set accordingly).
\emacroexp

\macroexp{\convertarray {<src>}{<dest>}}
Converts a normal array {\it src} to a macro array {\it dest}.
\emacroexp

\macroexp{\mergesort {<src>}{<dest>}}
Sorts the array {\it src} and places the result in {\it dest}.
\emacroexp

\subsubsection{Macro Arrays}

\macroexp{\createmarray {<name>}}
Creates a macro array by the name of {\it name}.
\emacroexp

\macroexp{\localizemarray {<name>}}
Localizes (see \gotoanchor{localization}{localization}) the macro array specified by {\it name}.
\emacroexp

\macroexp{\appendmarray {<name>}{<value>}}
Appends {\it value} to the macro array specified by {\it name}.
\emacroexp

\macroexp{\printmarray {<name>}}
Prints the macro array specifed by {\it name}.
\emacroexp

\macroexp{\convertmarray {<src>}{<dest>}}
Converts the macro array {\it src} into a normal array {\it dest}.
\emacroexp

\macroexp{\copymarray {<src>}{<dest>}}
Copies the macro array {\it src} into {\it dest}.
\emacroexp

\macroexp{\initmarray {<name>}{<x1>,...,<xN>}}
Creates a macro array {\it name} whose value is equivalent to $\tt[x1,\dots,xN]$.
\emacroexp

\macroexp{\findmarray {<name>}{<value>}<macro>}
Searches for {\it value} in the macro array {\it name}.
If found, sets \macro\@return@value{} to \macro\True{} and {\it macro} to the index where {\it value} was found.
Otherwise \macro\@return@value{} is set to \macro\False.
\emacroexp

\subsection{Stacks}

In the {\tt pdfData/stacks.tex} file, \pdftoolbox{} offers macros for creating and manipulating stack data structures.
There are two types of stacks, which differ in how they store their data.
They are generally used for different purposes:

\blist
    \item Normal stacks: these are normal stacks which store just the values given.
    \item Macro stacks: these stacks are meant to store only macros: they store both the definition and name of the macro.
\elist

\subsubsection{Normal Stacks}

\macroexp{\createstack {<name>}}
Creates a normal stack by the name of {\it name}.
\emacroexp

\macroexp{\stackpush {<name>}{<value>}}
Pushes the value {\it value} onto the stack specified by {\it name}.
\emacroexp

\macroexp{\stackdecrement {<name>}}
Pops from the top of the stack specified by {\it name} (deleting the value).
\emacroexp

\macroexp{\stackpop {<name>}<macro>}
Pops from the top of the stack specified by {\it name} into {\it macro}.
\emacroexp

\macroexp{\stacktop {<name>}<macro>}
Places the top of the stack specified by {\it name} into the macro {\it macro} without popping.
\emacroexp

\subsubsection{Macro Stacks} \anchor{macro stacks}

Macro stacks store macros, as opposed to values.
When pushing a macro \macro\X{} onto the stack, not only is the meaning of \macro\X{} stored, but so is its name.

\macroexp{\createmacrostack {<name>}}
Creates a macro stack by the name of {\it name}.
\emacroexp

\macroexp{\macrostackpush {<name>}<macro>}
Pushes the macro {\it macro} onto the macro stack specified by {\it name}.
\emacroexp

\macroexp{\macrostackdecrement {<name>}}
Pops from the top of the macro stack specified by {\it name} (deleting the value).
\emacroexp

\macroexp{\macrostackset {<name>}}
If the top of the macro stack specified by {\it name} has name \macro\X{} and value {\tt value}, sets \macro\X{} to {\tt value}.
\emacroexp

\macroexp{\macrostackpop {<name>}}
Pops from the top of the macro stack specified by {\it name} (same as \macro\macrostackset, but also pops the value off of the stack).
\emacroexp

\macroexp{\macrostackpeek {<name>}<macro1><macro2>}
If the top of the macro stack specified by {\it name} is $(\macro\X,{\tt value})$, then \macro\X{} is placed into {\it macro1}, and {\tt value} into {\it macro2}.
\emacroexp

\subsection{Localization} \anchor{localization}

Using \gotoanchor{macro stacks}{macro stacks}, \pdftoolbox{} allows for {\it localization}.
This gives the user the ability to create block scopes (as opposed to just plain-ol` \TeX{} groups).
The usage is simple and as follows:
\benum
    \item The user enters a scope using \macro\beginscope.
    \item The user {\it localizes} a macro \macro\X{} by doing \inlinecode|\localize\X|.
    \item The user exits the scope using \macro\endscope.
    Once the scope is exited, the previous definition of localized macros is restored.
\eenum
So for example,

\beginhi \color{white}
\def\X{0}
\beginscope
    \localize\X
    \def\X{1}
    \X
    \beginscope
        \def\X{2}
        \X
    \endscope
    \X
\endscope
\X
\endhi

Will output 1 2 2 0.
As opposed to

\beginhi \color{white}
\def\X{0}
\bgroup
    \def\X{1}
    \X
    \bgroup
        \def\X{2}
        \X
    \egroup
    \X
\egroup
\X
\endhi

Which will output 1 2 1 0.

\subsection{Counters}

In the {\tt pdfData/counters.tex}, \pdftoolbox{} implements counters.
Counters are simple wrappers over plain-\TeX{} counters.
They hold integer values, are mutable, and can be made dependent on one another so that when one is altered another is set to zero.

\macroexp{\createcounter {<name>}[<c1>,...,<cN>]}
Creates a counter by the name {\it name} dependent on counters {\it c1,\dots,cN}.
\emacroexp

\macroexp{\adddependentcounter {<secondary>}{<primary>}}
Makes the {\it secondary} counter dependent on the {\it primary} one; whenever {\it primary} is (non-independently; see e.g. \gotomacro\seticounter) altered, {\it secondary} is set to zero.
\emacroexp

\macroexp{\zerodependents {<primary>}}
Sets to zero all counters dependent on {\it primary}.
\emacroexp

\macroexp{\setcounter {<counter>}{<amount>}}
Sets {\it counter} to {\it amount} (zeroing all counters dependent on {\it counter}).
\emacroexp

\macroexp{\advancecounter {<counter>}{<amount>}}
Advances {\it counter} by {\it amount} (zeroing all counters dependent on {\it counter}).
\emacroexp

\macroexp{\seticounter {<counter>}{<amount>}}
Sets {\it counter} to {\it amount} ({\it without} zeroing all counters dependent on {\it counter}).
\emacroexp

\macroexp{\advanceicounter {<counter>}{<amount>}}
Advances {\it counter} by {\it amount} ({\it without} zeroing all counters dependent on {\it counter}).
\emacroexp

\macroexp{\counter {<name>}}
Returns the \TeX{} counter corresponding to the \pdftoolbox{} counter {\it name}.
Useful for example when printing the value of a counter: simply do \inlinecode|\the\counter{name}|.
\emacroexp

\subsection{Dictionaries}

In the {\tt pdfData/dictionaries.tex} file, \pdftoolbox{} implements dictionaries (also colloquially known as ``hashmaps'' or ``maps'').
These are simple maps between keys and values.

\macroexp{\createdict {<name>}}
Creates a dictionary by the name {\it name}.
\emacroexp

\macroexp{\adddict {<name>}{<key>}{<value>}}
Adds the $({\it key}:{\it value})$ key-value pair to the dictionary specified by {\it name}.
\emacroexp

\macroexp{\indexdict {<name>}{<key>}}
Expands to the value of {\it key} in the dictionary {\it name}.
\emacroexp

\macroexp{\keyindict {<name>}{<key>}}
Sets \macro\@return@value{} according to if {\it key} is found in the dictionary {\it name}.
\emacroexp

\subsection{Mappings}

In {\tt pdfData/key-value.tex}, \pdftoolbox{} implements the ability to pass key-value parameters to macros.

\macroexp{\mapkeys {<options>}{<input>}}
Maps the key-value pairs given in {\it input} according to {\it options}.
{\it options} is itself a set of key-value pairs, where the value of each key is an array which may contain:
\blist
    \item {\tt name} (required): the name of the macro to give the value of the key;
    \item {\tt required}: added if the key is required;
    \item {\tt definition}: what definition macro to use for defining the value (e.g. \macro\def, \macro\edef);
    \item {\tt mapping}: how to map the input to the value: the input is defined relative to {\tt definition} into a macro wrapped with {\tt mapping};
    \item {\tt default}: the default value of the key.
\elist
Or the value may be empty (no array), which means it is {\it valueless} and acts as a boolean flag.
\emacroexp

So for example, you may have a macro defined like so:

\beginhi \color{white}
\def\puthi#1{Hello (#1)}

\def\getinput#1{%
    \mapkeys{
        first={
            name=fst,
            required,
            definition=\edef,
            mapping=\puthi%
        },
        second={
            name=snd,
            default=S. Lurp%
        }%
    }{#1}%
}

\getinput{first=pdftoolbox}
(\fst) (\snd)
\endhi

This will output {\tt (Hello (pdftoolbox)) (S. Lurp)}.

\macroexp{\keyexists {<key>}<macro>\lastkeys}
This is an internal command, added to this documentation only due to its usefulness.
Given a key name {\it key}, this macro checks if it exists in the map corresponding to the last call to \macro\mapkeys{} (the macro itself is more versatile, but we restrict it to this case).
If the key does not exist, then {\it macro} is set to \macro\_nul.
This is useful with valueless keys.
\emacroexp

\bwarning
    \macro\mapkeys{} is a bit finnicky when it comes to spaces and commas, but the rule is simple: place a comment at the end of each list.
    That means that within each key's array, you must place a comment at the end (otherwise an extraneous space is added to the value), and after the last key's array you must place a comment.
\eppbox

