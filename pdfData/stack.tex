\input pdfToolbox-utils
\_checkloaded{stacks}

%   Creates a macro stack, a stack which stores the definition of macros.
%   #1: The name of the stack.
\def\createmacrostack#1{\_xp\def\csname mstack@#1@sp\endcsname{0}}

%   Gets the current pointer on a macro stack.
%   #1: The name of the stack.
\def\macrostackpointer#1{\csname mstack@#1@sp\endcsname}

%   Pushes a macro onto a macro stack.
%   #1: The name of the stack.
%   #2: The control sequence to push onto the stack.
\def\macrostackpush#1#2{%
    \_xp\edef\csname mstack@#1@sp\endcsname{\the\numexpr \csname mstack@#1@sp\endcsname + 1\relax}%
    \_xp\let\csname mstack@#1@def@\csname mstack@#1@sp\endcsname\endcsname=#2%
    \_xp\edef\csname mstack@#1@name@\csname mstack@#1@sp\endcsname\endcsname{\_mstrip#2}%
}

%   Gets a macro on a stack.
%   #1: The name of the stack
%   #2: The index on the stack
\def\macrostackget#1#2{%
    \csname mstack@#1@def@#2\endcsname%
}

%   Decrements the stack pointer of a macrostack without restoring the definition of the control sequence.
%   #1: The name of the stack.
\def\macrostackdecrement#1{%
    \_xp\let\csname mstack@#1@name@\csname mstack@#1@sp\endcsname\endcsname=\undefined%
    \_xp\let\csname mstack@#1@def@\csname mstack@#1@sp\endcsname\endcsname=\undefined%
    \_xp\edef\csname mstack@#1@sp\endcsname{\the\numexpr \csname mstack@#1@sp\endcsname - 1\relax}%
}

%   Sets the control sequence from the top of the stack, without popping it.
%   #1: The name of the stack
\def\macrostackset#1{%
    \_xplet{\csname\csname mstack@#1@name@\csname mstack@#1@sp\endcsname\endcsname\endcsname}{\csname mstack@#1@def@\csname mstack@#1@sp\endcsname\endcsname}%
}

%   Pops a control sequence off of the top of the stack.
%   This restores the definition of the control sequence.
%   #1: The name of the stack.
\def\macrostackpop#1{%
    \macrostackset{#1}%
    \macrostackdecrement{#1}%
}

%   Peeks at the top of a macro stack.
%   #1: The name of the stack.
%   #2: A control sequence to put the definition of the control sequence at the top of the stack into.
%   #3: A control sequence to put the name of the control sequence at the top of the stack into.
\def\macrostacktop#1#2#3{%
    \_xp\let#2\csname mstack@#1@def@\csname mstack@#1@sp\endcsname\endcsname%
    \_xp\let#3\csname mstack@#1@name@\csname mstack@#1@sp\endcsname\endcsname%
}

\def\createstack#1{\_xp\def\csname stack@#1@sp\endcsname{0}}

\def\stackpush#1#2{%
    \_xp\edef\csname stack@#1@sp\endcsname{\the\numexpr \csname stack@#1@sp\endcsname + 1\relax}%
    \_xp\def\csname stack@#1@\csname stack@#1@sp\endcsname\endcsname{#2}%
}

\def\stackdecrement#1{%
    \_xp\let\csname stack@#1@\csname stack@#1@sp\endcsname\endcsname=\undefined%
    \_xp\edef\csname stack@#1@sp\endcsname{\the\numexpr \csname stack@#1@sp\endcsname - 1\relax}%
}

\def\stackpop#1#2{%
    \stacktop{#1}{#2}%
    \stackdecrement{#1}%
}

\def\stacktop#1#2{%
    \_xp\let\_xp#2\csname stack@#1@\csname stack@#1@sp\endcsname\endcsname%
}

\def\_rstackpush#1#2{\stackpush{#2}{#1}}
\def\_rstackpop#1#2{\stackpop{#2}{#1}}

\createmacrostack{macro_stack}
\createstack{macro_stack_ptr}

\def\macro_stack_ptr{0}
\def\beginscope{%
    \_xp\_rstackpush\_xp{\macro_stack_ptr}{macro_stack_ptr}%
}

\def\endscope{%
    \stackpop{macro_stack_ptr}\_uregB%
    \loop\ifnum\macro_stack_ptr>\_uregB%
        \macrostackpop{macro_stack}%
        \edef\macro_stack_ptr{\the\numexpr \macro_stack_ptr - 1\relax}%
    \repeat%
}

\def\localize#1{%
    \macrostackpush{macro_stack}{#1}%
    \edef\macro_stack_ptr{\the\numexpr \macro_stack_ptr + 1\relax}%
}


